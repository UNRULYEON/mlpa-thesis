\styledchapter[Probleemanalyse]{probleemanalyse}

Zoals beschreven in \autoref{sec:aanleiding-opdracht} is NGTI genoodzaakt om \gls{machine-learning} in te zetten om haar applicaties 'slimmer' te maken. Hier zijn een aantal redenen voor, onder andere:
\begin{enumerate}
  \item gebruikerservaring verbeteren
  \item voorsprong hebben op concurrenten
\end{enumerate}

Het 'slimmer' maken van applicaties kan op verschillende manieren, maar met \gls{machine-learning} kan een platform gebouwd worden waarmee elke richting op gegaan kan worden. Om \gls{machine-learning} te implementeren in haar applicaties loopt NGTI tegen een aantal obstakels op, namelijk: expertise vereist in het \gls{machine-learning-pipeline} domein, tijd om een pipeline op te zetten en \gls{vendor-lock-in}.

\section{Expertise Machine Learning}\label{sec:expertise-machine-learning}
\Gls{machine-learning} is geen triviaal onderwerp. Om een model te trainen is kennis nodig van verschillende domeinen: \gls{data-mining}, software engineering en statistieken. In een multidisciplinair team is het voor één teamlid niet nodig om alle domeinen te beheersen.

Doordat er voorkennis nodig is om een model te trainen en een pipeline goed op te zetten, is het vaak te hoogdrempelig voor developers. De expertise is daar bovenop niet in een korte tijd te vergaren.

\section{Opzetten pipeline}\label{sec:opzetten-pipeline}
Bovenop de complexiteit van machine learning zelf bestaan er verschillende manieren om een model te trainen. Een \gls{machine-learning-pipeline} opzetten is daar een van. Een pipeline is een workflow dat bestaat uit een aantal stappen die doorgelopen kan worden om een model te trainen. In elke stap worden acties uitgevoerd, zoals het verwijderen van onbruikbare data of de prestatie van modellen vergelijken en een rapport met uitslagen genereren. Het opzetten van zo een pipeline én de actie(s) in de stappen definiëren kost tijd en kennis. Daarnaast zijn de stappen en acties vaak hetzelfde voor verschillende pipelines. Het automatiseren en hergebruiken van stappen en acties tussen pipelines zou tot tijdwinst leiden.

\section{Vendor lock-in}\label{sec:vendor-lock-in}
Er bestaan een aantal diensten, zogenoemde \acrfull{paas}, waarbij je een pipeline kan opzetten en acties kan definiëren. Een van de problemen met een \acrshort{paas} is \gls{vendor-lock-in}. Dit betekent dat, als er eenmaal een pipeline is opgezet, de overdraagbaarheid van de pipeline naar een andere \acrshort{paas} vrijwel onmogelijk is. Ook zijn de opties en mogelijkheden om uit te breiden in de toekomst gelimiteerd.

\section{Doelstelling}\label{sec:doelstelling}
Om het probleem op te lossen is onderzoek en experimentatie nodig op verschillende vlakken. De gewenste oplossing is het ontwikkelen van een systeem waarbij developers met weinig tot geen kennis een model kunnen trainen. Het systeem moet de infrastructurele taken voor zich nemen, zoals het opzetten van een pipeline, de stappen en acties automatiseren. Daarnaast moet het systeem ook platform agnostisch zijn.

\section{Bestaande oplossingen om pipelines op te zetten}\label{sec:bestaande-oplossingen-om-pipelines-op-te-zetten}
Er bestaan een aantal oplossingen voor vrijwel alle problemen die NGTI ondervindt. Elke oplossing is een \acrshort{paas} van een derde partij waarbij \gls{vendor-lock-in} inherent is. Dit maakt ze ongeschikt maar betekent echter niet dat ze nutteloos zijn. Er kan namelijk gekeken worden hoe een pipeline wordt opgezet, welke acties de stappen verricht en daar vervolgens (gedeeltelijk) het systeem op baseren. Daarnaast wordt bij alle oplossingen een expertise van ML op een bepaalde niveau verwacht.

De oplossingen kunnen gecategoriseerd worden in twee groepen:
\begin{enumerate}
  \item \Gls{machine-learning-pipeline} specifieke services
  \item \Glspl{cloud-computing-platform}
\end{enumerate}

\subsection{Machine learning pipeline specifieke service}\label{subsec:machine-learning-pipeline-specifieke-service}
Bedrijven zoals Algorithmia \cite{algorithmia-website} en Valohai \cite{valohai-website} bieden alleen diensten om pipelines op te zetten. Ze zorgen voor het databehoud dat door de gebruiker wordt geüpload en het trainen van het model. Verder kan er toezicht gehouden worden op de kosten, beschikbaarheid en prestatie van het model.

Valohai heeft documentatie een aantal blog posts die bij het ontwerpen van het systeem relevant zouden kunnen zijn.

\subsection{Cloud computing platformen}\label{subsec:cloud-computing-platformen}
De drie grote \glspl{cloud-computing-platform} Amazon, Azure en Google hebben meer te bieden dan alleen een pipeline opzetten, zoals het hosten van een website, database of virtuele server. De \glspl{cloud-computing-platform} hebben hetzelfde probleem als de \gls{machine-learning-pipeline} specifieke services; \gls{vendor-lock-in} is onvermijdelijk. Wat wel een mogelijkheid zou kunnen zijn is dat de andere services van de \glspl{cloud-computing-platform} gebruikt kunnen worden als onderdeel van het systeem.

Het systeem zou bijvoorbeeld een server kunnen aanmaken, een model trainen, het resultaat downloaden en vervolgens de server verwijderen.

% \subsection{Lokale frameworks}\label{subsec:lokale-frameworks}
% Naast de \glspl{cloud-computing-platform} en machine learning pipeline specifieke services bestaan er ook lokale frameworks om een model te trainen, zoals TensorFlow, Keras en PyTorch \cite{ml-libraries}. 

% \subsection{Algorithmia}\label{subsec:algorithmia}
% \subsection{Valohai}\label{subsec:valohai}
% \subsection{Azure Machine Learning Pipelines (Azure)}\label{subsec:azure-machine-learning-pipelines}
% \subsection{AI Platform Pipelines}\label{subsec:ai-platform-pipelines}
% \subsection{Amazon SageMaker Pipelines}\label{subsec:amazon-sagemaker-pipelines}

\section{Frameworks om cloud computing platformen te beheren}\label{sec:frameworks-om-cloud-computing-platformen-te-beheren}
Gedurende de vooronderzoek zijn frameworks dat cloud computing platformen beheert en frameworks waarmee een pipeline uitgerold kan worden naar boven gekomen. Deze frameworks kunnen deel uitmaken van de oplossing. Een framework dat cloud computing platformen kan beheren is Terraform \cite{terraform}. Met Terraform is het mogelijk om een plan te schrijven waarin bijvoorbeeld staat welke type server nodig is. Bij het uitvoeren van het plan spreekt Terraform een cloud computing platform naar keuze aan om de server op te starten. Terraform kan vervolgens controleren of de server draait en de server afsluiten wanneer het niet meer nodig is.

Kubeflow \cite{kubeflow} is een ander framework waarmee een pipeline kan worden opgezet. Het verschil met Terraform is dat Terraform flexibeler is met wat er aangemaakt kan worden op een cloud computing platform. Kubeflow kan alleen een pipeline uitrollen.

\section{Hoofd- en deelvragen}\label{sec:hoofd-en-deelvragen}
Uitgaand van de drie obstakels kan de hoofdvraag als volgt worden geformuleerd:

\begin{quoting}
  \begin{center}
    \textbf{
      \textit{
        In welke mate kan een machine learning pipeline worden geautomatiseerd onafhankelijk van de onderliggende cloud computing platform?
      }
    }
  \end{center}
\end{quoting}\smallskip

De hoofdvraag kan worden onderbouwd met vier deelvragen. Om te beginnen is het verstandig om te weten welke stappen er in een machine learning pipeline zit:

\begin{quoting}
  \begin{center}
    \textit{
      Uit welke stappen bestaat een machine learning pipeline?
    }
  \end{center}
\end{quoting}\smallskip

Vervolgens kunnen de verschillende cloud computing platformen in kaart worden gebracht:

\begin{quoting}
  \begin{center}
    \textit{
      Wat zijn de verschillen en overeenkomsten tussen cloud computing platformen waarmee een machine learning pipeline kan worden opgezet?
    }
  \end{center}
\end{quoting}\smallskip

Verder kan het handig zijn om meerdere platformen aan te spreken. Dit zou kunnen met een bestaand framework. Hierbij is het, net als de vorige deelvraag, belangrijk om te weten welke frameworks er zijn:

\begin{quoting}
  \begin{center}
    \textit{
      Wat zijn de verschillen en overeenkomsten tussen frameworks waarmee cloud computing platformen beheerd kunnen worden?
    }
  \end{center}
\end{quoting}\smallskip

Ten slotte wordt een PoC gemaakt om te laten zien of het probleem oplosbaar is. Hiervoor is een doordachte voorbereiden onmisbaar:

\begin{quoting}
  \begin{center}
    \textit{
      Hoe ziet de architecturale blauwdruk van een applicatie, waarmee een machine learning pipeline kan worden opgezet, die modulaire acties geautomatiseerd in stappen samenstelt, en die platform-onafhankelijk is, eruit?
    }
  \end{center}
\end{quoting}

% Uitgaand van de drie focuspunten in \autoref{sec:probleemdefinitie} kan de hoofdvraag als volgt worden geformuleerd:\smallskip

% \begin{quoting}
%   \begin{center}
%     \textbf{
%       \textit{
%         In welke mate kan een machine learning pipeline worden geautomatiseerd onafhankelijk van de onderliggende cloud computing platform?
%       }
%     }
%   \end{center}
% \end{quoting}\smallskip

% De hoofdvraag kan worden onderbouwd met vier deelvragen. Om te beginnen is het verstanding om te onderzoeken hoe een \gls{machine-learning} model wordt getraind:\smallskip

% \begin{quoting}
%   \begin{center}
%     \textit{
%       Welke stappen moeten worden ondernomen om een machine learning model te trainen?
%     }
%   \end{center}
% \end{quoting}\smallskip

% Vervolgens kan er op de deelvraag voortborduurt worden om het opzetten van een pipeline in kaart te brengen:\smallskip

% \begin{quoting}
%   \begin{center}
%     \textit{
%       Hoe wordt een machine learning pipeline opgezet?
%     }
%   \end{center}
% \end{quoting}\smallskip

% Verder zijn er verschillende \glspl{cloud-computing-platform} waarmee \gls{machine-learning} modellen getraind kunnen worden. Doordat het systeem platform agnostisch moet zijn is het van belang om verschillende platformen en frameworks te onderzoeken:

% \begin{quoting}
%   \begin{center}
%     \textit{
%       Wat zijn de verschillen en overeenkomsten tussen cloud computing platforms waarmee machine learning modellen kunnen worden getraind?
%     }
%   \end{center}
% \end{quoting}\smallskip

% Ten slotte wordt een PoC gemaakt om te laten zien of het probleem oplosbaar is. Hiervoor is een doordachte voorbereiden onmisbaar:\smallskip

% \begin{quoting}
%   \begin{center}
%     \textit{
%       Hoe ziet de architecturale blauwdruk van een applicatie, waarin een machine learning pipeline kan worden opgezet, die acties voorgeprogrammeerd zijn, en die platform-onafhankelijk is, eruit?
%     }
%   \end{center}
% \end{quoting}

% \newpage