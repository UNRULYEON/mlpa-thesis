\styledchapter[Onderzoeksmethoden en scope]{methode-van-onderzoek}

{\renewcommand{\arraystretch}{1.35}% for the vertical padding

Om elke hoofd- en deelvraag te beantwoorden, maak ik bij elk gebruik van een onderzoeksmethode. Volgens Scribbr \cite{research-methods} zijn er twee onderzoeksmethoden: kwantitatief en kwalitatief. Bij een kwantitatief onderzoeksmethode wordt data verzameld waarmee grafieken of tabellen gemaakt kunnen worden. De focus bij een kwalitatief onderzoeksmethode ligt bij het verzamelen van verschillende interpertaties en opvattingen. Hierop kan optioneel een eigen interpertatie op gemaakt worden. \cite{quantitative-vs-qualitative}.

Onder kwantitatief en kwalitatief vallen verschillende dataverzamelingsmethoden. Deze beschrijft simpelweg de manier hoe data wordt verzameld. Dit kan bijvoorbeeld met een enquête, literatuuronderzoek op websites en in boeken of een onderzoek over een lange periode \cite{quantitative-vs-qualitative}.

Elke hoofd- en deelvraag is gekoppeld aan een onderzoeksmethoden. Vervolgens is beschreven welk(e) dataverzamelingsmethode(n) wordt gebruikt met een korte toelichting. Daarnaast wordt op een hoog niveau de scope bepaald en de requirements vanuit NGTI mocht die er zijn.

\begin{table}[hbt!]
\centering
\begin{tabular}{|p{.215\linewidth}|p{.72\linewidth}|}
\hline
\multicolumn{2}{|p{.97\linewidth}|}{\textbf{D1: Uit welke stappen bestaat een machine learning pipeline?}} \\ \hline
  \textbf{Methode(s)}&
    Kwalitatief
  \\ \hline
  \textbf{Dataverzamelings methode(n)}&
    Literatuuronderzoek, fundamenteel onderzoek, toegepast onderzoek
  \\ \hline
  \textbf{Scope}&
    Binnen de scope:
    \begin{itemize}
      \item In kaart brengen uit welke stappen een pipeline bestaat
      \item Acties die in een stap worden uitgevoerd
      \item Of het mogelijk is om stappen te versimpelen / abstraheren voor developers
      \item Of het mogelijk is om stappen en acties te automatiseren
    \end{itemize}
    Buiten de scope:
    \begin{itemize}
      \item Automatisering van stappen en acties
      \item Een versimpeling van machine learning
    \end{itemize}
  \\ \hline
  \textbf{Toelichting}&
    Er wordt gekeken naar welke stappen er in een pipeline zitten. De theorie wordt vervolgens toegepast in een experiment. De nadruk ligt vooral of de mogelijkheid er is om stappen en acties te automatiseren en of machine learning versimpeld kan worden, niet dat er een uitwerking is.
  \\ \hline
\end{tabular}
\caption{Deelvraag 1}
\label{table:sq1}
\end{table}

\space
\newpage

\begin{table}[hbt!]
\centering
\begin{tabular}{|p{.215\linewidth}|p{.72\linewidth}|}
\hline
\multicolumn{2}{|p{.97\linewidth}|}{\textbf{D2: Wat zijn de verschillen en overeenkomsten tussen cloud computing platformen waarmee een machine learning pipeline kan worden opgezet?}} \\ \hline
  \textbf{Methode}&
    Kwalitatief
  \\ \hline
  \textbf{Dataverzamelings methode(n)}&
    Literatuuronderzoek, vergelijkend onderzoek
  \\ \hline
  \textbf{Scope}&
    Binnen de scope:
    \begin{itemize}
      \item Inventarisatie met de "long list short list"\space methode
      \item Welke functionaliteit bieden de platformen op een machine learning pipeline op te zetten
      \item Ervaring opdoen doormiddel van een pipeline te maken binnen twee cloud computing platformen
      \item Basale inventarisatie en vergelijking van alternatieve manieren om met een cloud computing platform te communiceren
    \end{itemize}
    Buiten de scope:
    \begin{itemize}
      \item Prijs, performance en snelheid
    \end{itemize}
  \\ \hline
  \textbf{Toelichting}&
    Cloud computing platformen worden in kaart gebracht. Vervolgens wordt met criteria bepaald in een later stadium de lijst verkort tot twee kandidaten. Alternatieve manieren om met een cloud computing platform te communiceren is afgebakend tot first-party tools en frameworks dat een of meerdere platformen tegelijk kan aanspreken.
  \\ \hline
\end{tabular}
\caption{Deelvraag 2}
\label{table:sq2}
\end{table}

\begin{table}[hbt!]
\centering
\begin{tabular}{|p{.215\linewidth}|p{.72\linewidth}|}
\hline
\multicolumn{2}{|p{.97\linewidth}|}{\textbf{D3: Wat zijn de verschillen en overeenkomsten tussen frameworks waarmee cloud computing platformen beheerd kunnen worden?}} \\ \hline
  \textbf{Methode}&
    Kwalitatief
  \\ \hline
  \textbf{Dataverzamelings methode(n)}&
    Literatuuronderzoek, vergelijkend onderzoek
  \\ \hline
  \textbf{Scope}&
    Binnen de scope:
    \begin{itemize}
      \item Inventarisatie naar wat er aangemaakt, gewijzigd en verwijderd kan worden binnen een cloud computing platform
      \item Hoe een machine learning pipeline op papier gemaakt zou worden met een framework
      \item Experiment met het opzetten van een pipeline via het framework op een cloud computing platform
    \end{itemize}
  \\ \hline
  \textbf{Toelichting}&
  Er wordt gekeken naar welke frameworks er beschikbaar zijn en wat de verschillen/overeenkomsten zijn. Om de applicatie future proof te maken is het belangrijk om een framework te kiezen wat in de praktijk beproefd is en ondersteuning van een community heeft.
  \\ \hline
\end{tabular}
\caption{Deelvraag 3}
\label{table:sq3}
\end{table}

\space
\newpage

\begin{table}[hbt!]
\centering
\begin{tabular}{|p{.215\linewidth}|p{.72\linewidth}|}
\hline
\multicolumn{2}{|p{.97\linewidth}|}{\textbf{D4: Hoe ziet de architecturale blauwdruk van een applicatie, waarin een machine learning pipeline kan worden opgezet, die acties voorgeprogrammeerd zijn, en die platform-onafhankelijk is, eruit?}} \\ \hline
  \textbf{Methode}&
    Kwalitatief
  \\ \hline
  \textbf{Dataverzamelings methode(n)}&
    Literatuuronderzoek
  \\ \hline
  \textbf{Scope}&
    Binnen de scope:
    \begin{itemize}
      \item Technische tekeningen
    \end{itemize}
    Buiten de scope:
    -
  \\ \hline
  \textbf{Toelichting}&
    De literatuuronderzoek slaat op of de technische tekeningen gemaakt zijn volgens een standaard zoals UML. Dit komt niet terug als theorie maar de bronnen worden wel vermeld.
  \\ \hline
\end{tabular}
\caption{Deelvraag 4}
\label{table:sq4}
\end{table}

\space
\newpage

\begin{table}[hbt!]
\centering
\begin{tabular}{|p{.215\linewidth}|p{.72\linewidth}|}
\hline
\multicolumn{2}{|p{.97\linewidth}|}{\textbf{H: In welke mate kan een machine learning pipeline worden geautomatiseerd onafhankelijk van de onderliggende cloud computing platform?}} \\ \hline
  \textbf{Methode}&
    Kwalitatief
  \\ \hline
  \textbf{Dataverzamelings methode(n)}&
    Literatuuronderzoek
  \\ \hline
  \textbf{Scope}&
    De scope wordt bepaald na de requirement analyse.
  \\ \hline
  \textbf{Toelichting}&
    Onderzoek naar documentatie van gebruikte framework(s).
  \\ \hline
\end{tabular}
\caption{Hoofdvraag}
\label{table:mq}
\end{table}