\styledchapter[Onderzoeksmethoden en scope]{methode-van-onderzoek}

\renewcommand{\arraystretch}{1.35} % for the vertical padding

Om elke hoofd- en deelvraag te beantwoorden, wordt er bij elk gebruik gemaakt van een onderzoeksmethode. Volgens Scribbr \cite{research-methods} zijn er twee onderzoeksmethoden: kwantitatief en kwalitatief. Bij een kwantitatief onderzoeksmethode wordt data verzameld waarmee bijvoorbeeld grafieken of tabellen gemaakt kunnen worden. De focus bij een kwalitatief onderzoeksmethode ligt bij het verzamelen van verschillende interpretaties en opvattingen. Hierop kan optioneel een eigen interpretaties op gemaakt worden. \cite{quantitative-vs-qualitative}.

Onder kwantitatief en kwalitatief vallen verschillende dataverzamelingsmethoden. Deze beschrijft simpelweg de manier hoe data wordt verzameld. Dit kan bijvoorbeeld met een enquête, literatuuronderzoek op websites en in boeken of een onderzoek over een lange periode \cite{quantitative-vs-qualitative}.

Elke hoofd- en deelvraag is gekoppeld aan een onderzoeksmethoden. Vervolgens is beschreven welk(e) dataverzamelingsmethode(n) wordt gebruikt met een korte toelichting. Daarnaast wordt op een hoog niveau de scope bepaald.

\begin{table}[hbt!]
  \centering
  \begin{tabular}{|p{.215\linewidth}|p{.72\linewidth}|}
  \hline
  \multicolumn{2}{|p{.97\linewidth}|}{\textbf{D1: Waar bestaat een machine learning pipeline uit?}} \\ \hline
    \textbf{Methode(s)}&
      Kwalitatief
    \\ \hline
    \textbf{Dataverzamelings methode(n)}&
      Literatuuronderzoek, fundamenteel onderzoek, toegepast onderzoek
    \\ \hline
    \textbf{Scope}&
      Bijlage \ref{appendix:scope-subquestion-1}
    \\ \hline
  \end{tabular}
  \caption{Onderzoeksmethode deelvraag 1}
  \label{table:research-method-subquestion-1}
\end{table}

\begin{table}[hbt!]
  \centering
  \begin{tabular}{|p{.215\linewidth}|p{.72\linewidth}|}
  \hline
  \multicolumn{2}{|p{.97\linewidth}|}{\textbf{D2: Hoe kan een orkestratietool verschillende cloud computing platformen beheren om een machine learning pipeline op te zetten?}} \\ \hline
    \textbf{Methode}&
      Kwalitatief
    \\ \hline
    \textbf{Dataverzamelings methode(n)}&
      Literatuuronderzoek, vergelijkend onderzoek
    \\ \hline
    \textbf{Scope}&
      Bijlage \ref{appendix:scope-subquestion-2}
    \\ \hline
  \end{tabular}
  \caption{Onderzoeksmethode deelvraag 2}
  \label{table:sq2}
\end{table}

\begin{table}[hbt!]
  \centering
  \begin{tabular}{|p{.215\linewidth}|p{.72\linewidth}|}
  \hline
  \multicolumn{2}{|p{.97\linewidth}|}{\textbf{D3: Hoe ziet de architecturale blauwdruk van een applicatie, waarmee een platform-onafhankelijk machine learning pipeline opgezet kan worden, eruit?}} \\ \hline
    \textbf{Methode}&
      Kwalitatief
    \\ \hline
    \textbf{Dataverzamelings methode(n)}&
      Literatuuronderzoek
    \\ \hline
    \textbf{Scope}&
      Bijlage \ref{appendix:scope-subquestion-3}
    \\ \hline
  \end{tabular}
  \caption{Onderzoeksmethode deelvraag 3}
  \label{table:sq3}
\end{table}

\space
\newpage

\begin{table}[hbt!]
  \centering
  \begin{tabular}{|p{.215\linewidth}|p{.72\linewidth}|}
  \hline
  \multicolumn{2}{|p{.97\linewidth}|}{\textbf{H: In welke mate kan een machine learning pipeline worden geautomatiseerd onafhankelijk van de onderliggende cloud computing platform?}} \\ \hline
    \textbf{Methode}&
      Kwalitatief
    \\ \hline
    \textbf{Dataverzamelings methode(n)}&
      Literatuuronderzoek
    \\ \hline
    \textbf{Scope}&
      Bijlage \ref{appendix:scope-main-question}
    \\ \hline
  \end{tabular}
  \caption{Onderzoeksmethode hoofdvraag}
  \label{table:mq}
\end{table}