\styledchapter[Discussie]{discussie}
Gedurende de scriptie is onderzoek gedaan naar ML pipeline, orkestratietools en hoe de architectuur er uitziet van een applicatie dat ML pipelines automatiseert. Daarnaast zijn een aantal experimenten uitgevoerd om de potentie van een ML pipeline en orkestratietool te valideren. 

De potentie van een ML pipeline is groter dan dat het lijkt. In \autoref{sec:ch4-advies} is uitgelegd hoe ML versimpeld kan worden voor de developer door extra stappen toe te voegen. Dit kan niet alleen gebruikt worden door NGTI maar ook andere bedrijven die geïnteresseerd zijn in het toepassen van ML. Een stap verder zou zijn dat een standaard wordt geïntroduceerd dat wordt toegepast door industrieleiders zoals Google, Microsoft en Azure. Momenteel hebben ze hun eigen variant wat betekent dat specifieke kennis voor één cloud platform nodig is. Verhuizen naar een ander cloud platform betekent essentieel dat de werking van een ander ML pipeline geleerd moet worden.

In \autoref{sec:ch7-advies} wordt er gesproken over hoe het definiëren van de configuratie vereenvoudigd kan worden door het te verdelen instappen. Vervolgens wordt er gevraagd wat voor type ML probleem het is en wat, op een hoog niveau, de pipeline moet doen. Dit zou niet alleen de PoC verbeteren, maar kan gebruikt worden als beginpunt voor developers nieuw zijn in het ML domein.