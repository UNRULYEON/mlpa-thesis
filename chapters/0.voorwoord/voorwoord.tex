\titleformat{\chapter}{\vspace{-1in}}{}{}{\Huge\textbf}
% \chapter*{Voorwoord}\thispagestyle{fancy}\vspace{-.35in}
\section*{Voorwoord}\thispagestyle{fancy}
Voor u ligt de scriptie "Machine Learning Pipeline Automation". Deze scriptie is geschreven in het kader van mijn afstuderen aan de opleiding Informatica aan de Hogeschool Rotterdam en in opdracht van het stagebedrijf NGTI. De afstudeerstage liep van februari 2021 tot juni 2021.

De onderzoeksvraag is bedacht door mijn bedrijfsbegeleider Okke van 't Verlaat. Halverwege de afstudeerstage heeft Kolja van der Vaart de bedrijfsbegeleiding overgenomen. Terugkijkend was de vraag complex en ambitieus. Ik heb me bezig gehouden met machine learning, automatisering en cloud platformen. 

Bij deze wil ik mijn afstudeerbegeleiders Okke van 't Verlaat, Kolja van der Vaart en vanuit school Stelian Paraschiv en Marian Slingerland bedanken voor de zorgvuldige begeleiding. Ik heb met Okke en Kolja effectief kunnen sparren over het onderzoek. Mijn begeleiders hebben mij ook geholpen om de scriptie in de juiste richting te sturen. Tot slot wil ik mijn vrienden bedanken voor het bieden van een ander perspectief om deze scriptie.

Ik wens u veel leesplezier toe.

Amar Kisoensingh

Zoetermeer, \today

\newpage

\section*{Samenvatting}
Een probleem met machine learning (ML) is dat het gebruik ervan tijd en kennis vereist. Ook is er kans op vendor lock-in als ML wordt gebruikt op een cloud platform zoals Google Cloud of Microsoft Azure. Het is dus vrijwel onmogelijk om te verhuizen naar een andere cloud platform.

Het doel van deze scriptie is om niet alleen te onderzoeken in hoeverre het mogelijk is om ML te automatiseren, maar ook of dit kan ongeacht de cloud platform. De onderzoeksvraag kan als volgt opgesteld worden: kan ML geautomatiseerd worden op een cloud platform-agnostisch wijze.

Om deze vraag te beantwoorden is er onderzoek gedaan naar ML pipelines. Hiervoor is literatuuronderzoek en experimentatie van pas gekomen. Met ML pipelines kan op een gestructureerd manier gewerkt worden met ML. Dit verlaagt de leercurve en maakt ML toegankelijker voor nieuwe developers. Naast het onderzoek is ook een proof of concept (PoC) gebouwd om de platform-agnostisch kant te valideren. Met het onderzoek en experimentatie dat gedaan is blijkt dat het mogelijk is om ML te automatiseren op een cloud platform-agnostisch manier.

Het onderzoek kan verder uitgebreid worden door te kijken naar hoe ML toegankelijker gemaakt kan worden voor developers en hoe het systeem robuuster opgezet kan worden.

\newpage

\section*{Summary}
One problem with machine learning (ML) is that using it requires time and knowledge. There is also a chance of vendor lock-in if ML is being used on a cloud platform such as Google Cloud or Microsoft Azure. It is virtually impossible to move to another cloud platform.

The aim of this thesis is not only to investigate to what extent ML can be automated, but also whether this is possible regardless of the cloud platform. The research question can be formulated as follows: can ML be automated in a cloud platform agnostic way.

To answer this question, research has been done on ML pipelines. Literature research and experimentation were utilized. With ML pipelines you can work with ML in a structured way. This not only makes ML more accessible, but lowers the learning curve for new developers. In addition to the research, a proof of concept (PoC) has also been built to validate the platform-agnostic aspect. The research and experimentation that has been done shows that it is possible to automate ML in a cloud platform agnostic way.

The research can be further expanded by looking at how ML can be made more accessible for developers and how the system can be set up more robustly. 

\newpage