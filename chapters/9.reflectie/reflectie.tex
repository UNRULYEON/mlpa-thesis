\styledchapter[Reflectie]{reflectie}
Het afstudeerproces is ook een groot leerproces voor mij geweest. De behandelde materie was niet alleen nieuw maar ook zeer complex. Daarnaast heb ik een aantal fouten gemaakt waardoor ik in tijdsnood kwam.

In de eerste helft van de afstudeerstage lag de focus te veel op ML. Ik deed ontzettend veel onderzoek naar ML omdat ik weinig tot niks wist van het onderwerp. Gedurende het onderzoek was ik theoretisch bezig en programmeerde of experimenten ik niet; er was geen balans tussen theoretisch en praktisch werk. In de initiële planning was de tijd gelijkmatig over elk deel- en hoofdvraag verdeeld (bijlage \ref{appendix:global-planning}). Naarmate er werd gewerkt aan de scriptie nam de theorie over ML langzaam de planning over en schoof de overige deel- en hoofdvraag verder op om meer ruimte te maken. Het onderzoek was gelukkig niet voor niets. De stappen in een pipeline worden genomen omdat deze nodig zijn om een model te trainen. Door de kennis die ik heb opgedaan werd dat snel duidelijk en begreep ik wat er gebeurde in een pipeline.

Na een stap terug te hebben gedaan werd het duidelijk dat ML pipelines de focus was, en niet ML zelf. Halverwege de afstudeerstage heb ik samen met mijn bedrijfsbegeleiders gezeten om de focus en scope te bepalen. Hierdoor werd het snel duidelijk waarnaar ik onderzoek moest doen en wat ik moest bouwen. Dit heeft enorm geholpen met mijn productiviteit. Door de duidelijke grens van wat wel en niet binnen de scope viel deed ik geen onderzoek naar onderwerpen die niet relevant waren. Dit is achteraf gezien een grote maar niet duidelijke valkuil in het ML domein. Er zijn talloze algoritmes met elk hen eigen theorie en er zijn ontzettend veel manieren om dingen te doen. Terugkijkend heb ik stukken geschreven die niet geschreven hoefde te worden en ook niet in deze scriptie zijn beland.

Een andere valkuil was de manier waarop ik aan de scriptie werkte. Doordat ik vooral theoretisch bezig was had ik moeite met de structuur van hoofdstukken en de samenhang tussen paragrafen. Ook halverwege heb ik besloten met zowel de school- als bedrijfsbegeleiders dat ik praktischer bezig zou zijn en wat minder tijd aan de scriptie zou besteden. Het praktisch bezig zijn hield voor mij in dat er meer en sneller experimenten werden gedaan. In de scriptie zette ik dan steekwoorden/zinnen om de kern van een alinea te definiëren. Via deze manier kon ik de structuur van een hoofdstuk goed bepalen en kon er weer theoretisch gewerkt worden.

Iets wat ik vanaf het begin had moeten doen was het maken van een Trello bord om de status van taken bij te houden. Nadat ik een Trello bord had ingericht voor de laatste vier weken om aan de PoC te werken, wist ik dat een Trello bord beter zou werken voor mij. Bij elke meeting (45 in totaal) heb ik genotuleerd, maar uiteindelijk heb ik niet veel met de notities gedaan (bijlage \ref{appendix:meeting-notes}). Met een Trello bord zou ik suggesties en andere taken die in een meeting voorbij schieten kunnen organiseren. Ook dient het als een herinnering zodat ik taken niet vergeet te doen. Tijdens de laatste vier weken ben ik ook begonnen met het gebruik van de Pomodorotechniek. Dit is een werkwijze wat helpt om de focus bij het schrijven  van de scriptie te houden. Met deze techniek maak je een lijst met taken. Daarna start je een timer en na 25 minuten neem je een pauze van 5 minuten. Dit doe je vier keer waarna je een pauze kan nemen van 15 minuten. Ikzelf heb pomofocus.io gebruikt. Met deze website kan je inschatten hoeveel tijd een taak kost. Vervolgens geeft pomofocus een verwachte eindtijd.

Voor een volgende scriptie zou ik een Trello bord inrichten voor taken en gebruik maken van de Pomodorotechniek om de focus bij het werk te houden. Notuleren tijdens meetings heeft niet veel zin en suggesties/taken kunnen gelijk in het Trello bord. Daarnaast is de focus en scope belangrijk voor de scriptie; zonder dit val ik in de diepte en verspil ik tijd.

% W7 gerealiseerd dat de focus te veel ligt op ML en niet op een ml pipeline. Te veel de diepte ingedoken met de theorie. Daarom ervoor gekozen om de deelvragen te wijzigen en op een hoger niveau te werken. 

% W8 - W14
% Ik loop achter volgens de planning doordat ik de focus te veel op het verslag heb gelegd. Er is veel research gedaan en weinig programmeren. De deelvraag over de cloud computing platformen is weggehaald en de deelvraag over de frameworks is versimpeld. 

% W9 - W15

% 12-05-2021
% Tickets in Trello waren niet goed geprioriteerd. High priority tickets werden gedaan na tickets met medium prioriteiten omdat de tickets met high prioriteiten een dependency hadden op medium tickets. Dit is aangepast zodat de tickets met high als eerst worden gedaan, vervolgens de tickets met medium en als laatst de tickets met low.

% Ook waren de taken niet goed verdeeld over de sprints.

% Hoofdvraag

% Al snel bleek dat de IaC framework Pulumi niet aan de eisen voldeed als er met meer dan een cloud platform gewerkt werd. Het aanmaken van componenten in een cloud platform was niet consistent tussen de twee cloud platformen. Bij Google Cloud werden componenten vrijwel gelijk beschikbaar na het versturen van de opdracht om het aan te maken. In vergelijking met Azure was dit niet het geval. Het framework Pulumi gaf wel aan dat het bij Azure gelukt was, maar de componenten waren niet gelijk na het aanmaken beschikbaar. Door deze verschil in gedrag kan de terugkoppeling van Pulumi niet vertrouwd worden. Daarnaast waren er problemen met het aanmaken van componenten binnen Azure. Vaak kwamen verschillende errors naar boven zonder dat er iets gewijzigd was. 

% Omdat de stuk code van Pulumi in de PoC een klein onderdeel was, werd er gekeken naar alternatieven. Doordat andere IaC frameworks dezelfde of ander onverwacht gedrag kan vertonen, is er voor gekozen om de frameworks van cloud platformen zelf te gebruiken. De documentatie van de frameworks is uitgebreid en, volgens de documentatie, is het mogelijk om alle acties te verrichtten die de PoC moet doen. Het vervangen van Pulumi was simpel omdat de keuze in het begin was gemaakt en de hoeveelheid code dat vervangen moest worden vrij weinig was.

% Waarom niet andere frameworks?
% Voor elke platform hadden we platform specifiek code en het werkte niet lekker
% Door pulumi het generiek idee tegen het licht gehouden. Waarom dit want het werkt niet en andere werken ook niet. Daarom specifieke platform package. Terraform is ook niet geschikt. Wat lost pulumi op en waarom lost het niet mijn probleem op? Het is de verkeerde tool.