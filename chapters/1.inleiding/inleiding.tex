\styledchapter[Inleiding]{inleiding}

\section{Het bedrijf: NGTI}
NGTI is mobile app development bureau gevestigd in Rotterdam, Nederland. Zij maakt hoogwaardige mobiele applicaties voor native, hybrid en webgebruik.

\subsection{Klanten van NGTI}


\subsection{Tools die worden gebruikt}
Om productief te zijn gebruikt NGTI een aantal tools en programma's om producten te maken en te communiceren met zowel collega's als klanten.

\subsubsection{Slack}
Interne communicatie gaat via Slack. Het programma facaliteert collega's om elkaar met een lage instap te benaderen en berichten die voor het hele bedrijf gelden te versturen. Ook zijn er 'channels' beschikbaar over specifieke onderwerpen, zoals: \textit{\#dev}, \textit{\#ios} en \textit{\#test-automation}.

\subsubsection{Google Workspace}
Met Google Workspace kunnen bestanden en documenten gemaakt, opgeslagen en gedeeld worden. Omdat dit via een browser kan, hoeven werknemers geen software te installeren. NGTI gebruikt het ook om collaboratief en parallel te werken aan hetzelfde document.

\subsubsection{Zoom}
\hl{Voorheen werdt Zoom alleen gebruikt om te videobellen met collega's en interviewees.} In de tijd van het pandemie is Zoom echter een belangrijke speler geworden om effectief samen te werken. Meetings zoals introducties van nieuwe collega's of demo's van producten worden online gehouden.

\section{Opdracht}
NGTI heeft voorzien dat ze haar applicaties 'slimmer' moet maken door \gls{machine-learning} in te zetten. Niet alleen zorgt dit voor een verhoogde \textit{user experience}, maar geeft NGTI ook een voorsprong op haar concurrenten.

Om \gls{machine-learning} toe te passen is het raadzaam om een \gls{machine-learning-pipeline} op te zetten. Een pipeline is een gestructureerde werkwijze om een model te trainen. Het opzetten van de pipeline en een competente model trainen is tijdrovend en vereist kennis in het domein. Vaak worden modellen getrained op een \gls{cloud-computing-platform} zoals Azure, AWS of Google Cloud. Het probleem met platformen zoals deze is dat het ontzettend lastig is om te wisselen van platform.\bigskip\bigskip\bigskip\bigskip\bigskip

De opdracht bestaat uit twee onderdelen:
\begin{enumerate}
  \item onderzoek naar het automatiseren van het opzetten van een \gls{machine-learning-pipeline} om het laagdrempelig en minder tijdrovend te maken.
  \item onderzoek naar het maken van een platform agnostische oplossing
\end{enumerate}

Hierbij zal een PoC gemaakt worden om aan te tonen of het haalbaar is. Een diepere duik in het probleem is te vinden in \autoref{chap:probleemanalyse}.

\section{Leeswijzer}
