\styledchapter[Onderzoeksmethoden]{methode-van-onderzoek}

{\renewcommand{\arraystretch}{1.35}% for the vertical padding

\begin{table}[hbt!]
\centering
\begin{tabular}{|l|p{.83\linewidth}|}
\hline
\multicolumn{2}{|p{.97\linewidth}|}{\textbf{D1: Welke stappen moeten worden ondernomen om een machine learning-model te trainen?}} \\ \hline
  \textbf{Doel}&
    Onderzoek naar wat machine learning is en hoe een model getraind kan worden. Een visualisatie van hoe een model getraind wordt, een PoC waarbij een model getraind wordt en een conclusie en advies met vooral waarop gelet moet worden bij het trainen.
  \\ \hline
  \textbf{Methode}&
    Kwalitatief
  \\ \hline
  \textbf{Validatie}&
    Er is uitleg gegeven wat machine learning is en hoe een model getraind kan worden op een simpele manier waarbij de visualisatie de uitleg compleet maakt. Met het conclusie en advies kan iemand die vrijwel geen kennis heeft aan de slag met het trainen van een model.
  \\ \hline
\end{tabular}
\caption{Deelvraag 1}
\label{table:sq1}
\end{table}

\begin{table}[hbt!]
\centering
\begin{tabular}{|l|p{.83\linewidth}|}
\hline
\multicolumn{2}{|p{.97\linewidth}|}{\textbf{D2: Hoe wordt een machine learning pipeline opgezet?}} \\ \hline
  \textbf{Doel}&
    Onderzoek naar wat een machine learning pipeline is, waarom het gebruikt wordt en welke onderdelen van de pipeline geautomatiseerd kunnen worden. PoC waarbij een pipeline met het model van de vorige deelvraag wordt opgezet. Conclusie en advies.
  \\ \hline
  \textbf{Methode}&
    Kwalitatief
  \\ \hline
  \textbf{Validatie}&
    Het is duidelijk wat een pipeline is en waarom het handig/noodzakelijk is om te gebruiken. In de conclusie en advies is het onder andere duidelijk of onderdelen geautomatiseerd kunnen worden.
  \\ \hline
\end{tabular}
\caption{Deelvraag 2}
\label{table:sq2}
\end{table}

\begin{table}[hbt!]
\centering
\begin{tabular}{|l|p{.83\linewidth}|}
\hline
\multicolumn{2}{|p{.97\linewidth}|}{\textbf{D3: Wat zijn de verschillen en overeenkomsten tussen cloud computing platforms en lokale frameworks waarmee machine learning-modellen kunnen worden getraind?}} \\ \hline
  \textbf{Doel}&
    De grote cloud computing platformen en lokale frameworks in kaart brengen. De features, voor- en nadelen worden met elkaar vergeleken. Waar van toepassing wordt een kosten berekening gedaan. Conclusie en advies bevat onder andere welke platform en/of framework het beste is om mee te beginnen.
  \\ \hline
  \textbf{Methode}&
    Kwalitatief
  \\ \hline
  \textbf{Validatie}&
    Een overzichtelijke vergelijking tussen de cloud computing platformen en lokale frameworks is gemaakt. Uit de conclusie en advies moet blijken welke platform en/of framework het beste is om mee te beginnen.
  \\ \hline
\end{tabular}
\caption{Deelvraag 3}
\label{table:sq3}
\end{table}

\space
\newpage

\begin{table}[hbt!]
\centering
\begin{tabular}{|l|p{.83\linewidth}|}
\hline
\multicolumn{2}{|p{.97\linewidth}|}{\textbf{D4: Hoe ziet de architecturale blauwdruk van een applicatie, waarin een machine learning pipeline kan worden opgezet en die platform-onafhankelijk is, eruit?}} \\ \hline
  \textbf{Doel}&
    Er wordt research gedaan naar hoe een pipeline geautomatiseerd opgezet kan worden en naar verschillende tools/frameworks. Technische designs worden gemaakt ter validatie en als blauwdruk.
  \\ \hline
  \textbf{Methode}&
    Kwalitatief
  \\ \hline
  \textbf{Validatie}&
    Uit de research en designs is het duidelijk hoe een systeem gebouwd kan worden.
  \\ \hline
\end{tabular}
\caption{Deelvraag 4}
\label{table:sq4}
\end{table}

\begin{table}[hbt!]
\centering
\begin{tabular}{|l|p{.83\linewidth}|}
\hline
\multicolumn{2}{|p{.97\linewidth}|}{\textbf{H: Hoe kan een machine learning pipeline worden geautomatiseerd onafhankelijk van de onderliggende cloud computing platform of lokale framework?}} \\ \hline
  \textbf{Doel}&
    Uit de requirements wordt de scope bepaald. Er worden designs gemaakt om de interface te testen. Een PoC wordt gebouwd.
  \\ \hline
  \textbf{Methode}&
    Kwalitatief
  \\ \hline
  \textbf{Validatie}&
    De PoC is werkzaam en voldoet aan de requirements.
  \\ \hline
\end{tabular}
\caption{Hoofdvraag}
\label{table:mq}
\end{table}