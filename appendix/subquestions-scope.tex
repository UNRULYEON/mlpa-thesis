\section{Scope deelvraag 1}\label{appendix:scope-subquestion-1}
\begin{table}[hbt!]
  \centering
  \begin{tabular}{|p{.215\linewidth}|p{.72\linewidth}|}
  \hline
  \multicolumn{2}{|p{.97\linewidth}|}{\textbf{D1: Uit welke stappen bestaat een machine learning pipeline?}} \\ \hline
    \textbf{Scope}&
      Binnen de scope:
      \begin{itemize}
        \item In kaart brengen uit welke stappen een pipeline bestaat
        \item Acties die in een stap worden uitgevoerd
        \item Of het mogelijk is om stappen te versimpelen / abstraheren voor developers
        \item Of het mogelijk is om stappen en acties te automatiseren
      \end{itemize}
      Buiten de scope:
      \begin{itemize}
        \item Automatisering van stappen en acties
        \item Een versimpeling van machine learning
      \end{itemize}
    \\ \hline
    \textbf{Toelichting}&
      Er wordt gekeken naar welke stappen er in een pipeline zitten. De theorie wordt vervolgens toegepast in een experiment. De nadruk ligt vooral of de mogelijkheid er is om stappen en acties te automatiseren en of machine learning versimpeld kan worden, niet dat er een uitwerking is.
    \\ \hline
  \end{tabular}
  \caption{Scope deelvraag 1}
  \label{table:scope-subquestion-1}
\end{table}

\newpage

\section{Scope deelvraag 2}\label{appendix:scope-subquestion-2}
\begin{table}[hbt!]
  \centering
  \begin{tabular}{|p{.215\linewidth}|p{.72\linewidth}|}
  \hline
  \multicolumn{2}{|p{.97\linewidth}|}{\textbf{D2: Hoe kan een orkestratietool verschillende cloud computing platformen beheren om een machine learning pipeline op te zetten?}} \\ \hline
    \textbf{Scope}&
      Binnen de scope:
      \begin{itemize}
        \item High-level uitleg over hoe een orkestratietool werkt
        \item Inventarisatie met de "knock-out"\space methode
        \item Criteria lijst voor orkestratietools
        \item Opmerkelijke features die relevant zijn voor de PoC
        \item Ervaring opdoen doormiddel van een pipeline te maken op twee cloud computing platformen
      \end{itemize}
      Buiten de scope:
      \begin{itemize}
        \item Performance en snelheid
      \end{itemize}
    \\ \hline
    \textbf{Toelichting}&
      Frameworks dat cloud computing platformen beheert worden in kaart gebracht. Met knock-out criteria wordt de lijst verkort. Met een framework wordt een pipeline opgezet.
    \\ \hline
  \end{tabular}
  \caption{Scope deelvraag 2}
  \label{table:scope-subquestion-2}
  \end{table}

\section{Scope deelvraag 3}\label{appendix:scope-subquestion-3}
\begin{table}[hbt!]
  \centering
  \begin{tabular}{|p{.215\linewidth}|p{.72\linewidth}|}
  \hline
  \multicolumn{2}{|p{.97\linewidth}|}{\textbf{D3: Hoe ziet de architecturale blauwdruk van een applicatie, waarmee een platform-onafhankelijk machine learning pipeline opgezet kan worden, eruit?}} \\ \hline
    \textbf{Scope}&
      Binnen de scope:
      \begin{itemize}
        \item Technische tekeningen
      \end{itemize}
      Buiten de scope:
      -
    \\ \hline
    \textbf{Toelichting}&
      De literatuuronderzoek slaat op of de technische tekeningen gemaakt zijn volgens een standaard zoals UML. Dit komt niet terug als theorie maar de bronnen worden wel vermeld.
    \\ \hline
  \end{tabular}
  \caption{Scope deelvraag 3}
  \label{table:scope-subquestion-3}
\end{table}

\newpage

\section{Scope hoofdvraag}\label{appendix:scope-main-question}
\begin{table}[hbt!]
  \centering
  \begin{tabular}{|p{.215\linewidth}|p{.72\linewidth}|}
  \hline
  \multicolumn{2}{|p{.97\linewidth}|}{\textbf{H: In welke mate kan een machine learning pipeline worden geautomatiseerd onafhankelijk van de onderliggende cloud computing platform?}} \\ \hline
    \textbf{Scope}&
      De scope wordt bepaald na de requirement analyse.
    \\ \hline
    \textbf{Toelichting}&
      Onderzoek naar documentatie van gebruikte framework(s).
    \\ \hline
  \end{tabular}
  \caption{Scope hoofdvraag}
  \label{table:scope-main-question}
\end{table}

\newpage

% \section*{Scope hoofd- en deelvragen}\label{appendix:scope-subquestions-and-main-question}

% \subsection{Scope deelvraag 1}\label{appendix:scope-subquestion-1}
% \begin{table}[hbt!]
%   \centering
%   \begin{tabular}{|p{.215\linewidth}|p{.72\linewidth}|}
%   \hline
%   \multicolumn{2}{|p{.97\linewidth}|}{\textbf{D1: Uit welke stappen bestaat een machine learning pipeline?}} \\ \hline
%     \textbf{Scope}&
%       Binnen de scope:
%       \begin{itemize}
%         \item In kaart brengen uit welke stappen een pipeline bestaat
%         \item Acties die in een stap worden uitgevoerd
%         \item Of het mogelijk is om stappen te versimpelen / abstraheren voor developers
%         \item Of het mogelijk is om stappen en acties te automatiseren
%       \end{itemize}
%       Buiten de scope:
%       \begin{itemize}
%         \item Automatisering van stappen en acties
%         \item Een versimpeling van machine learning
%       \end{itemize}
%     \\ \hline
%     \textbf{Toelichting}&
%       Er wordt gekeken naar welke stappen er in een pipeline zitten. De theorie wordt vervolgens toegepast in een experiment. De nadruk ligt vooral of de mogelijkheid er is om stappen en acties te automatiseren en of machine learning versimpeld kan worden, niet dat er een uitwerking is.
%     \\ \hline
%   \end{tabular}
%   \caption{Scope deelvraag 1}
%   \label{table:scope-subquestion-1}
% \end{table}

% \newpage

% \subsection{Scope deelvraag 2}\label{appendix:scope-subquestion-2}
% \begin{table}[hbt!]
%   \centering
%   \begin{tabular}{|p{.215\linewidth}|p{.72\linewidth}|}
%   \hline
%   \multicolumn{2}{|p{.97\linewidth}|}{\textbf{D2: Hoe kan een orkestratietool verschillende cloud computing platformen beheren om een machine learning pipeline op te zetten?}} \\ \hline
%     \textbf{Scope}&
%       Binnen de scope:
%       \begin{itemize}
%         \item High-level uitleg over hoe een orkestratietool werkt
%         \item Inventarisatie met de "knock-out"\space methode
%         \item Criteria lijst voor orkestratietools
%         \item Opmerkelijke features die relevant zijn voor de PoC
%         \item Ervaring opdoen doormiddel van een pipeline te maken op twee cloud computing platformen
%       \end{itemize}
%       Buiten de scope:
%       \begin{itemize}
%         \item Performance en snelheid
%       \end{itemize}
%     \\ \hline
%     \textbf{Toelichting}&
%       Frameworks dat cloud computing platformen beheert worden in kaart gebracht. Met knock-out criteria wordt de lijst verkort. Met een framework wordt een pipeline opgezet.
%     \\ \hline
%   \end{tabular}
%   \caption{Scope deelvraag 2}
%   \label{table:scope-subquestion-2}
%   \end{table}

% \subsection{Scope deelvraag 3}\label{appendix:scope-subquestion-3}
% \begin{table}[hbt!]
%   \centering
%   \begin{tabular}{|p{.215\linewidth}|p{.72\linewidth}|}
%   \hline
%   \multicolumn{2}{|p{.97\linewidth}|}{\textbf{D3: Hoe ziet de architecturale blauwdruk van een applicatie, waarmee een platform-onafhankelijk machine learning pipeline opgezet kan worden, eruit?}} \\ \hline
%     \textbf{Scope}&
%       Binnen de scope:
%       \begin{itemize}
%         \item Technische tekeningen
%       \end{itemize}
%       Buiten de scope:
%       -
%     \\ \hline
%     \textbf{Toelichting}&
%       De literatuuronderzoek slaat op of de technische tekeningen gemaakt zijn volgens een standaard zoals UML. Dit komt niet terug als theorie maar de bronnen worden wel vermeld.
%     \\ \hline
%   \end{tabular}
%   \caption{Scope deelvraag 3}
%   \label{table:scope-subquestion-3}
% \end{table}

% \newpage

% \subsection{Scope hoofdvraag}\label{appendix:scope-main-question}
% \begin{table}[hbt!]
%   \centering
%   \begin{tabular}{|p{.215\linewidth}|p{.72\linewidth}|}
%   \hline
%   \multicolumn{2}{|p{.97\linewidth}|}{\textbf{H: In welke mate kan een machine learning pipeline worden geautomatiseerd onafhankelijk van de onderliggende cloud computing platform?}} \\ \hline
%     \textbf{Scope}&
%       De scope wordt bepaald na de requirement analyse.
%     \\ \hline
%     \textbf{Toelichting}&
%       Onderzoek naar documentatie van gebruikte framework(s).
%     \\ \hline
%   \end{tabular}
%   \caption{Scope hoofdvraag}
%   \label{table:scope-main-question}
% \end{table}

% \newpage