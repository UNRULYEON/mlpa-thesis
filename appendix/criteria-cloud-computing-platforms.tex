\section*{Criteria met toelichting voor cloud computing platformen}\label{appendix:criteria-with-elaboration-for-cloud-computing-platforms}

\begin{table}[hbt!]
  \centering
  \caption{Criteria voor cloud computing platformen}
  \begin{tabular}{|p{.2\linewidth}|p{.69\linewidth}|}
  \hline
  \textbf{Criteria} & \textbf{Toelichting} \\ \hline
    ML Pipeline \newline aanmaken
    &
    Het platform moet het aanmaken van een machine learning pipeline kunnen ondersteunen. 
    \\ \hline

    Serverbeheer
    &
    Aanmaken, wijzigen en verwijderen van een server. Een server kunnen aanmaken bij het platform is praktisch omdat een server voor meerdere doeleinde gebruikt kan worden.
    \\ \hline

    Database beheer
    &
    Aanmaken, wijzigen en verwijderen van een database. De PoC moet data ergens kunnen opslaan en vandaan halen; dit is mogelijk in een database.
    \\ \hline

    Storage \newline bucket beheer
    &
    Aanmaken, wijzigen en verwijderen van een storage bucket. In een storage bucket kan grote hoeveelheden data opgeslagen worden zoals bestanden en afbeeldingen. Dit kan handig zijn om bijvoorbeeld train en test data op te slaan.
    \\ \hline

    Uptime 99.9\%
    &
    Het platform moet een vorm van uptime kunnen garanderen waarbij het percentage zo dicht mogelijk bij 99.9 ligt. Hierdoor is de kans dat NGTI door een storing bij een platform niet productief kan zijn zo klein mogelijk.
    \\ \hline

    Regionale \newline beschikbaarheid
    &
    Waar data wordt opgeslagen en de servers draaien is vrij belangrijk. Het platform moet ten minste West-Europa ondersteunen in verband met de gegevensbescherming in de EU (GDPR). In een ideale situatie zou het platform ook specifiek Zwitserland ondersteunen aangezien de meeste klanten van NGTI vandaar komen.
    \\ \hline

    Toegankelijkheid documentatie ML pipeline
    &
    In het geval dat de PoC wordt uitgebreid tot een applicatie is het belangrijk om onderhoud uit te voeren en eventueel nieuwe functies toe te voegen. Het is daarom van belang dat het platform documentatie heeft over ML pipelines. 
    \\ \hline

    APIs
    &
    Om het platform programmatisch aan te sturen zijn APIs dat het platform beschikbaar stelt onmisbaar. Op deze manier kunnen frameworks het platform beheren en kan eventueel een op maat gemaakte oplossing gebruikt worden.
    \\ \hline

    Inhoud ML
    &
    Voor NGTI is het belangrijk dat het platform niet alleen ondersteuning heeft om modellen te trainen, maar meer mogelijkheden biedt zoals het schrijven van een eigen algoritme of out-of-the-box oplossingen.
    \\ \hline
  \end{tabular}
  \label{table:criteria-cloud-computing-platforms}
\end{table}

\newpage